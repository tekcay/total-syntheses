
\twocolumn[{%
\maketitle
\thispagestyle{first}
\vspace{-30pt}
\begin{mybox}
\textcolor{Blue}{\textbf{ASBTRACT:}} A new and innovative retrosynthesis of ($-$)-Himandrine from commercially available products is proposed in 18 steps. In response to today\textquoteright{}s challenges in total synthesis, this synthetic route is entirely carried out without any transition metal, and very few protecting groups are used. These guidelines showcase some of the recent advances in organocatalysis, and prove that organometallic coupling are  not necessary to obtain a natural product. Known and efficient reactions are proposed to obtain ($-$)-Himandrine, such as an inverse electron-demand Diels-Alder and enamine/iminium catalysis. The synthetic route also features an interesting 10-member ring contraction. Even though the efficiency of this approach is not assessed, it is hoped that this will inspire organic chemists to focus on strategy rather than tactics to build complex molecular structures.
\end{mybox}
}]

%----------------------------------
\lettrine[lines=2,loversize=0.1]{\textcolor{Blue}{T}}{ }his alkaloid belongs to the \textit{Galbulimima} family, a class of compounds that can be found in the bark of \textit{Galbulimima belgraveana}, a tree located in Australia and Papua New Guinea. Himandrine is a weak atropine-like agent causing relaxation of spasms induced by the choline esters acetylcholine and carbamyl choline. Even though its activity is quite weak, further tests can be made to assess all of its bio-active properties. It is therefore necessary to synthesize this molecule in an efficient way.\par
%----------------------------------
%----------------------------------
\begin{figure''}
\centering
    \replacecmpd[TMP1]{Himandrine1.one}
	\replacecmpd[TMP2]{Himandrine1.two}
	\replacecmpd[TMP3]{Himandrine1.three}
		\includegraphics[scale=0.8]{data/Structures.eps}	
\captionof{figure}{($-$)-Himandrine structures}	
    \label{fig:structures}
\end{figure''}
%----------------------------------
%----------------------------------
The only total synthesis of ($-$)-Himandrine has been reported by Movassaghi and coworkers in 2009\autocite{Movassaghi09}. Its structure has been represented in three different ways (\ref{fig:structures}), by Movassaghi [\cmpd{Himandrine1.one}], by O'Connor [\cmpd{Himandrine1.two}]\autocite{OConnor10} and by Larson et al. [\cmpd{Himandrine1.three}]\autocite{Larson15}. It contains a decaline pattern A and B, one piperidine E and a [3.2.1]-octane system C, D and F and possesses ten stereogenic centers (C-5, C-6, C-9, C-11, C-12, C-14, C-15, C-18, C-19, C-21) and one tertiary amine (\ref{fig:number}).\\
%----------------------------------
%----------------------------------
\begin{figure''}
\centering
		\includegraphics[scale=0.8]{data/Himandrine_number.eps}
	\captionof{figure}{($-$)-Himandrine carbon number attribution}
    \label{fig:number}
\end{figure''}
%----------------------------------
%----------------------------------
The goal of that project was to propose an elegant route to this molecule, using the least amount of protecting groups and avoiding transition metals such as palladium. These ground rules are established to follow green chemistry principles. They certainly made the retrosynthetic analysis challenging but interesting nonetheless: classical reactions did not bring a decent rendering according to our taste, and no precedent in the literature supported all our approaches. Hereby it led us to thinking outside of the box and conceiving a bold route.
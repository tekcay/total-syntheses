\section*{CONCLUSION}
\addcontentsline{toc}{chapter}{Conclusion}
We herein report a new retrosynthetic approach to ($-$)-Himandrine featuring a 10-member ring lactamisation, a ring contraction reaction and an inverse electron-demand Diels Alder. The synthesis is metal-free, counts 18 steps which only few were dedicated to protection and deprotection. As Phil Baran said\autocite{Baran09} : \textquoteright{}\textit{They are a necessary evil sometimes, but I think that ever since we and others started putting it forward that you could consciously exclude protecting groups from the way that synthetic designs are formulated and executed, there\textquoteright{}s been a large number of synthesis and methodologies saying, let\textquoteright{}s see what happens when we leave the protecting groups out.}\textquoteright{} The same precept can be transposed to metals.\\
However seeking for the least amount of steps is not always the best choice. It might be better to plan more steps that are more reliable and afford higher yield. Besides, the way of counting steps can be debatable aswell. A one-pot procedure is counted as one step even though many transformations occur and even solvent or secondary products can be removed. It is yet commonly acknowledged that what should separate each step is a purification step such as chromatography column. That is the reason why the number of steps can dramatically change following one\textquoteright{}s perspective.\\
In summary, the proposed pathway relies on quite robust reactions but it is obvious that use of transition metal coupling would have permitted an easier, shorter and probably more efficient synthesis. Nonetheless the pursuit of a metal/protecting group-free route ideality should be a constant source of motivation, as freeing ourselves from what we have aldready seen over and over again stimulates our imagination and creativity.